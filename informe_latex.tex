\documentclass[12pt,a4paper]{article}
\usepackage[utf8]{inputenc}
\usepackage[spanish]{babel}
\usepackage{geometry}
\usepackage{graphicx}
\usepackage{listings}
\usepackage{xcolor}
\usepackage{hyperref}
\usepackage{fancyhdr}
\usepackage{titlesec}
\usepackage{enumitem}
\usepackage{amsmath}
\usepackage{booktabs}
\usepackage{longtable}

% Configuración de página
\geometry{margin=2.5cm}
\pagestyle{fancy}
\fancyhf{}
\rhead{Santiago Ibarra}
\lhead{Proyecto Backend Node.js}
\rfoot{Página \thepage}

% Configuración de código
\lstset{
    language=JavaScript,
    basicstyle=\ttfamily\small,
    keywordstyle=\color{blue},
    commentstyle=\color{green!60!black},
    stringstyle=\color{red},
    numbers=left,
    numberstyle=\tiny,
    frame=single,
    breaklines=true,
    showstringspaces=false
}

% Configuración de títulos
\titleformat{\section}{\Large\bfseries}{\thesection}{1em}{}
\titleformat{\subsection}{\large\bfseries}{\thesubsection}{1em}{}

\begin{document}

% Portada
\begin{titlepage}
    \centering
    \vspace*{2cm}
    
    {\Huge\bfseries Proyecto de Implementación de Sitios Web Dinámicos\par}
    \vspace{1cm}
    
    {\Large\bfseries ACTIVIDAD 11: Instalación, Configuración y Utilización de Node.js para Backend con JavaScript\par}
    \vspace{2cm}
    
    {\Large\bfseries Sistema de Gestión de Tareas (Todo App)\par}
    \vspace{2cm}
    
    \begin{minipage}{0.4\textwidth}
        \begin{flushleft}
            \large
            \textbf{Materia:} Diseño e Implementación de Sitios Web Dinámicos\\
            \textbf{Escuela:} EEST N.º 1 - "Eduardo Ader" Vicente López\\
            \textbf{Curso:} 7° Año 2° Grupo A\\
            \textbf{Día y horario:} Miércoles de 17:35 a 21:45 Hs\\
            \textbf{Modalidad:} Presencial/Virtual Asincrónico
        \end{flushleft}
    \end{minipage}
    \hfill
    \begin{minipage}{0.4\textwidth}
        \begin{flushright}
            \large
            \textbf{Profesor:} Jorge Fabián Siles Guzmán\\
            \textbf{Estudiante:} Santiago Ibarra\\
            \textbf{1° Cuatrimestre 2025}
        \end{flushright}
    \end{minipage}
    
    \vfill
    {\large \today\par}
\end{titlepage}

\tableofcontents
\newpage

\section{Introducción}

Este informe documenta la implementación completa de un sistema backend utilizando Node.js, Express.js y MySQL para la gestión de tareas (Todo App). El proyecto fue desarrollado como parte de la Actividad 11 de la materia "Diseño e Implementación de Sitios Web Dinámicos" en la EEST N.º 1 "Eduardo Ader".

El sistema implementa funcionalidades completas de autenticación de usuarios, gestión de tareas con operaciones CRUD, y una API RESTful bien estructurada. Se incluye también un frontend básico para interactuar con el backend.

\section{Link Github}



\section{Arquitectura del Sistema}

\subsection{Estructura del Proyecto}

El proyecto sigue una arquitectura MVC (Model-View-Controller) con la siguiente estructura:



\subsection{Tecnologías Utilizadas}

\begin{itemize}
    \item \textbf{Node.js}: Runtime de JavaScript para el servidor
    \item \textbf{Express.js}: Framework web para Node.js
    \item \textbf{MySQL}: Base de datos relacional
    \item \textbf{MySQL2}: Driver de MySQL para Node.js
    \item \textbf{bcryptjs}: Encriptación de contraseñas
    \item \textbf{jsonwebtoken}: Autenticación JWT
    \item \textbf{cors}: Middleware para CORS
    \item \textbf{dotenv}: Gestión de variables de entorno
    \item \textbf{nodemon}: Reinicio automático en desarrollo
\end{itemize}

\section{Configuración del Proyecto}

\subsection{Dependencias del Proyecto}

El archivo \texttt{package.json} contiene las siguientes dependencias:

\begin{lstlisting}
{
  "name": "trabajon12",
  "version": "1.0.0",
  "main": "app.js",
  "scripts": {
    "test": "echo \"Error: no test specified\" && exit 1",
    "dev": "nodemon app.js",
    "start": "node app.js"
  },
  "dependencies": {
    "bcryptjs": "^3.0.2",
    "cors": "^2.8.5",
    "dotenv": "^17.2.0",
    "express": "^5.1.0",
    "jsonwebtoken": "^9.0.2",
    "mysql2": "^3.14.2"
  },
  "devDependencies": {
    "nodemon": "^3.1.10"
  }
}
\end{lstlisting}

\subsection{Variables de Entorno}

El archivo \texttt{entorn.env} contiene la configuración de la base de datos:

\begin{lstlisting}
DB_HOST=localhost
DB_USER=root
DB_PASSWORD=
DB_NAME=todo_app
PORT=3000
JWT_SECRET=tu_secreto_jwt_super_seguro
\end{lstlisting}

\section{Implementación del Backend}

\subsection{Configuración del Servidor Principal}

El archivo \texttt{app.js} configura el servidor Express con todos los middlewares necesarios:

\begin{lstlisting}
const express = require('express');
const cors = require('cors');
require('dotenv').config({ path: './entorn.env' });

const authRoutes = require('./routes/authRoutes');
const taskRoutes = require('./routes/taskRoutes');

const app = express();

// Middlewares
app.use(cors());
app.use(express.json());
app.use(express.static('public'));

// Health check route
app.get('/api/health', (req, res) => {
    res.json({ status: 'OK', message: 'Server is running' });
});

// Rutas
app.use('/api/auth', authRoutes);
app.use('/api/tasks', taskRoutes);

const PORT = process.env.PORT || 3000;

app.listen(PORT, () => {
    console.log(`Servidor corriendo en http://localhost:${PORT}`);
});
\end{lstlisting}

\subsection{Configuración de la Base de Datos}

El archivo \texttt{config/db.js} establece la conexión con MySQL:

\begin{lstlisting}
const mysql = require('mysql2');
require('dotenv').config();

const pool = mysql.createPool({
    host: process.env.DB_HOST,
    user: process.env.DB_USER,
    password: process.env.DB_PASSWORD,
    database: process.env.DB_NAME,
    waitForConnections: true,
    connectionLimit: 10,
    queueLimit: 0
});

module.exports = pool.promise();
\end{lstlisting}

\section{Modelos de Datos}

\subsection{Modelo de Usuario}

El archivo \texttt{models/user.js} define las operaciones de base de datos para usuarios:

\begin{lstlisting}
const db = require('../config/db');

class User {
    static async create(username, email, password) {
        const sql = 'INSERT INTO users (username, email, password) VALUES (?, ?, ?)';
        const [result] = await db.execute(sql, [username, email, password]);
        return result.insertId;
    }

    static async findByEmail(email) {
        const sql = 'SELECT * FROM users WHERE email = ?';
        const [rows] = await db.execute(sql, [email]);
        return rows[0];
    }

    static async findById(id) {
        const sql = 'SELECT id, username, email, created_at FROM users WHERE id = ?';
        const [rows] = await db.execute(sql, [id]);
        return rows[0];
    }
}

module.exports = User;
\end{lstlisting}

\subsection{Modelo de Tarea}

El archivo \texttt{models/task.js} define las operaciones CRUD para tareas:

\begin{lstlisting}
const db = require('../config/db');

class Task {
    static async create(title, description, userId) {
        const sql = 'INSERT INTO tasks (title, description, user_id) VALUES (?, ?, ?)';
        const [result] = await db.execute(sql, [title, description, userId]);
        return result.insertId;
    }

    static async findByUserId(userId) {
        const sql = 'SELECT * FROM tasks WHERE user_id = ? ORDER BY created_at DESC';
        const [rows] = await db.execute(sql, [userId]);
        return rows;
    }

    static async findById(id) {
        const sql = 'SELECT * FROM tasks WHERE id = ?';
        const [rows] = await db.execute(sql, [id]);
        return rows[0];
    }
    
    static async update(id, title, description, completed) {
        const sql = 'UPDATE tasks SET title = ?, description = ?, completed = ? WHERE id = ?';
        const [result] = await db.execute(sql, [title, description, completed, id]);
        return result.affectedRows;
    }

    static async delete(id) {
        const sql = 'DELETE FROM tasks WHERE id = ?';
        const [result] = await db.execute(sql, [id]);
        return result.affectedRows;
    }
}

module.exports = Task;
\end{lstlisting}

\section{Controladores}

\subsection{Controlador de Autenticación}

El archivo \texttt{controllers/authController.js} maneja el registro y login de usuarios:



\subsection{Controlador de Tareas}

El archivo \texttt{controllers/taskController.js} implementa las operaciones CRUD para tareas:



\section{Middlewares}

\subsection{Middleware de Autenticación}

El archivo \texttt{middlewares/authMiddleware.js} verifica tokens JWT:



\section{Rutas}

\subsection{Rutas de Autenticación}

El archivo \texttt{routes/authRoutes.js} define los endpoints de autenticación:

\begin{lstlisting}
const express = require('express');
const router = express.Router();
const authController = require('../controllers/authController');

router.post('/register', authController.register);
router.post('/login', authController.login);

module.exports = router;
\end{lstlisting}

\subsection{Rutas de Tareas}

El archivo \texttt{routes/taskRoutes.js} define los endpoints protegidos para tareas:

\begin{lstlisting}
const express = require('express');
const router = express.Router();
const taskController = require('../controllers/taskController');
const authMiddleware = require('../middlewares/authMiddleware');

router.get('/', authMiddleware, taskController.getAllTasks);
router.post('/', authMiddleware, taskController.createTask);
router.put('/:id', authMiddleware, taskController.updateTask);
router.delete('/:id', authMiddleware, taskController.deleteTask);

module.exports = router;
\end{lstlisting}

\section{API Endpoints}

\subsection{Endpoints de Autenticación}

\begin{longtable}{|l|l|l|p{6cm}|}
\hline
\textbf{Método} & \textbf{Endpoint} & \textbf{Descripción} & \textbf{Parámetros} \\
\hline
POST & \texttt{/api/auth/register} & Registrar nuevo usuario & \texttt{\{username, email, password\}} \\
\hline
POST & \texttt{/api/auth/login} & Iniciar sesión & \texttt{\{email, password\}} \\
\hline
\end{longtable}

\subsection{Endpoints de Tareas (Protegidos)}

\begin{longtable}{|l|l|l|p{6cm}|}
\hline
\textbf{Método} & \textbf{Endpoint} & \textbf{Descripción} & \textbf{Parámetros} \\
\hline
GET & \texttt{/api/tasks} & Obtener todas las tareas del usuario & Header: \texttt{Authorization: Bearer <token>} \\
\hline
POST & \texttt{/api/tasks} & Crear nueva tarea & \texttt{\{title, description\}} + Header: \texttt{Authorization} \\
\hline
PUT & \texttt{/api/tasks/:id} & Actualizar tarea existente & \texttt{\{title, description, completed\}} + Header: \texttt{Authorization} \\
\hline
DELETE & \texttt{/api/tasks/:id} & Eliminar tarea & Header: \texttt{Authorization} \\
\hline
\end{longtable}

\subsection{Endpoints de Sistema}

\begin{longtable}{|l|l|l|p{6cm}|}
\hline
\textbf{Método} & \textbf{Endpoint} & \textbf{Descripción} & \textbf{Respuesta} \\
\hline
GET & \texttt{/api/health} & Verificar estado del servidor & \texttt{\{status: "OK", message: "Server is running"\}} \\
\hline
\end{longtable}

\section{Frontend}

\subsection{Interfaz de Usuario}

El proyecto incluye un frontend básico en \texttt{public/index.html} que permite:

\begin{itemize}
    \item Registro e inicio de sesión de usuarios
    \item Creación, edición y eliminación de tareas
    \item Visualización de tareas en tiempo real
    \item Interfaz responsive y moderna
\end{itemize}

\subsection{Estilos CSS}

Los estilos están definidos en \texttt{public\_css/style.css} con un diseño moderno y responsive.

\subsection{JavaScript del Cliente}

La lógica del frontend está en \texttt{public\_js/main.js} e incluye:

\begin{itemize}
    \item Gestión de tokens JWT
    \item Llamadas a la API REST
    \item Manipulación del DOM
    \item Validación de formularios
\end{itemize}

\section{Base de Datos}

\subsection{Esquema de la Base de Datos}

El sistema utiliza MySQL con las siguientes tablas:

\subsubsection{Tabla users}
\begin{lstlisting}
CREATE TABLE users (
    id INT PRIMARY KEY AUTO_INCREMENT,
    username VARCHAR(50) UNIQUE NOT NULL,
    email VARCHAR(100) UNIQUE NOT NULL,
    password VARCHAR(255) NOT NULL,
    created_at TIMESTAMP DEFAULT CURRENT_TIMESTAMP
);
\end{lstlisting}

\subsubsection{Tabla tasks}
\begin{lstlisting}
CREATE TABLE tasks (
    id INT PRIMARY KEY AUTO_INCREMENT,
    title VARCHAR(255) NOT NULL,
    description TEXT,
    completed BOOLEAN DEFAULT FALSE,
    user_id INT NOT NULL,
    created_at TIMESTAMP DEFAULT CURRENT_TIMESTAMP,
    updated_at TIMESTAMP DEFAULT CURRENT_TIMESTAMP ON UPDATE CURRENT_TIMESTAMP,
    FOREIGN KEY (user_id) REFERENCES users(id) ON DELETE CASCADE
);
\end{lstlisting}

\section{Instalación y Configuración}

\subsection{Requisitos Previos}

\begin{enumerate}
    \item Node.js (versión 14 o superior)
    \item MySQL (versión 5.7 o superior)
    \item npm (incluido con Node.js)
\end{enumerate}

\subsection{Pasos de Instalación}

\begin{enumerate}
    \item \textbf{Clonar o descargar el proyecto}
    \item \textbf{Instalar dependencias:}
    \begin{lstlisting}
    npm install
    \end{lstlisting}
    
    \item \textbf{Configurar la base de datos:}
    \begin{lstlisting}
    CREATE DATABASE todo_app;
    USE todo_app;
    \end{lstlisting}
    
    \item \textbf{Ejecutar los scripts SQL para crear las tablas}
    
    \item \textbf{Configurar variables de entorno} en \texttt{entorn.env}
    
    \item \textbf{Ejecutar el servidor:}
    \begin{lstlisting}
    npm run dev
    \end{lstlisting}
\end{enumerate}

\section{Pruebas y Validación}

\subsection{Pruebas de la API}

Se realizaron pruebas exhaustivas de todos los endpoints:



\subsection{Errores Solucionados}

Durante el desarrollo se identificaron y corrigieron los siguientes errores:

\begin{enumerate}
    \item \textbf{Errores de sintaxis:} Eliminación de comentarios \texttt{[cite: ...]} inválidos
    \item \textbf{Problemas de case-sensitive:} Corrección de imports de archivos
    \item \textbf{Configuración de variables de entorno:} Especificación correcta de la ruta del archivo
    \item \textbf{Rutas faltantes:} Implementación de endpoint de health check
\end{enumerate}

\section{Características de Seguridad}

\subsection{Autenticación JWT}

\begin{itemize}
    \item Tokens con expiración de 1 hora
    \item Verificación automática en rutas protegidas
    \item Manejo seguro de errores de autenticación
\end{itemize}

\subsection{Encriptación de Contraseñas}

\begin{itemize}
    \item Uso de bcryptjs con salt rounds de 10
    \item Almacenamiento seguro de contraseñas en la base de datos
    \item Comparación segura durante el login
\end{itemize}

\subsection{Validación de Datos}

\begin{itemize}
    \item Validación de campos obligatorios
    \item Sanitización de entradas de usuario
    \item Manejo de errores de base de datos
\end{itemize}

\section{Optimizaciones Implementadas}

\subsection{Base de Datos}

\begin{itemize}
    \item Pool de conexiones para mejor rendimiento
    \item Índices en campos de búsqueda frecuente
    \item Consultas optimizadas con prepared statements
\end{itemize}

\subsection{Código}

\begin{itemize}
    \item Estructura modular y reutilizable
    \item Manejo asíncrono de operaciones
    \item Separación clara de responsabilidades (MVC)
\end{itemize}

\section{Conclusiones}

Este proyecto demuestra la implementación exitosa de un backend completo utilizando Node.js y Express.js. Se han logrado todos los objetivos planteados en la consigna académica:



El sistema desarrollado es escalable, mantenible y sigue las mejores prácticas de desarrollo web moderno. La arquitectura MVC implementada permite una fácil extensión de funcionalidades y la separación clara de responsabilidades facilita el mantenimiento del código.

\section{Anexos}

\subsection{Comandos Útiles}


\subsection{Recursos Adicionales}

\begin{itemize}
    \item \href{https://nodejs.org/}{Documentación oficial de Node.js}
    \item \href{https://expressjs.com/}{Documentación oficial de Express.js}
    \item \href{https://www.mysql.com/}{Documentación oficial de MySQL}
    \item \href{https://jwt.io/}{Información sobre JWT}
\end{itemize}

\end{document} 